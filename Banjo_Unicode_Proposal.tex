\documentclass[12pt]{article}

\usepackage{url}
\usepackage{cite}
\usepackage{hyperref}

% define the title
\author{Tyler H. Cipriani\\
  \texttt{tyler@tylercipriani.com}}
\title{Proposal for the encoding of a banjo glyph}

\begin{document}
  \maketitle

  % Create abstract named "Proposal"
  \renewcommand{\abstractname}{Proposal}

  \begin{abstract}
    The Unicode Consortium proposes adding 1 UCS character to further the goal of completeness
    for the set of glyphs represented on the SMP block: Miscellaneous Symbols and Pictographs\cite{scherer}.
    The current ``emoji" block, set to establish interoperability of existing code-sets
    with the UCS standard, is incomplete in several key areas. This addition of
    a banjo glyph to an empty character space in this code block does nothing to
    disrupt previously established goals of this block, but, rather, extends this
    block to include a larger community of customers.

    There is also a demonstrated demand for the proposed glyph among a community of
    customers in bluegrass and banjo discussion groups in online communities.
  \end{abstract}

  \section[Proposed Character: Category and Description]{Proposed Character:\\ Category and Description}
    \begin{itemize}
      \item \textbf{Category}: B.1-Specialized (small collection)
      \item \textbf{Category}: Other Symbols (So)
      \item \textbf{Block}: Miscellaneous Symbols and Pictographs
      \item \textbf{Character Name}: `BANJO'
      \item \textbf{Glyph}:  U+1F3DB
    \end{itemize}

    \subsection{Description of Usage}
      Glyph would be used in any place it would be appropriate to represent 
      banjo music or the invocation of what the idea of ``banjo" represents.
      This includes, but is not limited to, representing bluegrass music,
      the earliest recorded popular music, early American music traditions, or
      African music traditions (as this is the region from where the banjo is
      believed to have originated)\cite{allen}.

  \section{Justifications: category and name}
    \subsection{Name}
    As glyph is a banjo, the name `BANJO' accurately represents, and is 
    the accepted name for, that which is shown in the glyph.

    \subsection{Category}
    The category to which the symbol is assigned is part of the rational for
    adding the glyph--set completeness. The block and general category already
    house the glyphs for `MUSICAL KEYBOARD', `TRUMPET', `SAXOPHONE' and `VIOLIN'. 
    `BANJO' would be added to this block and group for set completeness.

  \section{Groups and `Community of Customers'}
    \begin{enumerate}
      \item \href{http://www.banjohangout.org/topic/278863}{Banjo Hangout discussion forum}
    \end{enumerate}

  \begin{thebibliography}{1}
    \bibitem{allen} Greg Allen {\em The Banjo's Roots, Reconsidered } \August 23, 2011: All Things Considered. NPR News. \url{http://www.npr.org/2011/08/23/139880625/the-banjos-roots-reconsidered}

    \bibitem{scherer} Markus Scherer {\em Proposal for Encoding Emoji Symbols } \March 5, 2009. \url{https://sites.google.com/site/unicodesymbols/Home/emoji-symbols/proposal-text}
  \end{thebibliography}

\end{document}
